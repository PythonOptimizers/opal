\documentclass{report}

\usepackage{tikz}

%----------------------

\title{Analysis and Design Document \\ for \\ OPAL}
\begin{document}
\maketitle
\chapter{Introduction}
\par The primary goal is create a framework that helps the users to realize the tunning task following the schema

\begin{figure}[htpb]
  \centering
  \usetikzlibrary{shapes,arrows} 
  
  % Define block styles
  \tikzstyle{decision} = [diamond, draw, fill=blue!20, 
  text width=4.5em, text badly centered, node distance=3cm, inner sep=0pt]
  \tikzstyle{data} = [draw, ellipse,fill=red!20, node distance=3cm,
  minimum height=2em]
  \tikzstyle{line} = [draw, -latex']
  \tikzstyle{computing} = [rectangle, draw, fill=blue!20, 
  text width=5em, text centered, rounded corners, minimum height=4em] 

  \begin{tikzpicture}[node distance = 2cm, auto]
    % Place nodes
    \node [computing] (algorithm) {Target solver};
    \node [data,left of=algorithm] (test-prob) {Test problem};
    \node [data, right of=algorithm] (param) {Parameter};
    \node [data, below of=algorithm] (measure-table) {Table of measures};
    \node [computing, below of=measure-table] (evaluator) {Measure evaluator};
    \node [data, left of=evaluator, node distance=3.5cm] (model) {Evaluating model};
    \node [data, below of=evaluator] (evaluating-results) {Evaluating results};
    \node [computing, right of=evaluator, node distance=3cm] (solver) {Direct search solver};
    % Draw edges
    \path [line] (algorithm) -- (test-prob);
    \path [line] (param) -- (algorithm);
    \path [line] (algorithm) -- (measure-table);
    \path [line] (measure-table) -- (evaluator);
    \path [line] (model) -- (evaluator);
    \path [line] (evaluator) -- (evaluating-results);
    \path [line,dashed] (evaluating-results) -| (solver);
    \path [line,dashed] (solver) -- (param);
\end{tikzpicture}
  \caption{General schema of parameter tunning}
  \label{fig:parameter-tunning-schema}
\end{figure}
\chapter{Backgrounds}
\par The principles are built basing on the observations:
\begin{itemize}
  \item There are main entities Information and Information Manipulator
    \begin{enumerate}
    \item Information is in fact set of elements with the methods set and get value. The 
      set is organized in the different structure like a scalar, a vector or a matrix ...
    \item Information Manipulator represent for the processes manipulate the input information to
      get the other information as output
    \end{enumerate}
  \item Any process is formulated by the interaction between these entities
  \item An Information Manipulator is characterized by Input, Parameters and Output.
    \begin{enumerate}
    \item Input includes the Information and set of Manipulators that may 
      be a empty set. If this set is empty, the Manipulator is called Evaluator, otherwise
      it is called Solver. 
    \item Parameter is actually Information, it is used to generallized a class of Manipulator.
      Each time the parameters are set to the specific values, we have a manipulator.
    \item Output is Information represents for the results of manipulation.
    \end{enumerate}
\end{itemize}

\chapter{System analysis}
\chapter{System design}
\chapter{}
\end{document}